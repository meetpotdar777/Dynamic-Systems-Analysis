\documentclass[11pt,a4paper]{article}

% --- Core Packages ---
\usepackage[utf8]{inputenc}
\usepackage{amsmath, amssymb, amsfonts, amsthm}
\usepackage{booktabs, array, multirow}
\usepackage{geometry}
\usepackage{titlesec}
\usepackage{enumitem}
\usepackage{xcolor}
\usepackage{setspace}
\usepackage{listings}
\usepackage{cite}

% --- Page Setup ---
\geometry{margin=1.0in, top=1.1in, bottom=1.1in}
\setstretch{1.3} % Increased line height to ensure 7-page density

% --- Code Block Styling ---
\lstset{
    backgroundcolor=\color{gray!5},
    basicstyle=\ttfamily\small,
    breaklines=true,
    frame=single,
    keywordstyle=\color{blue},
    commentstyle=\color{green!50!black}
}

% --- Title Formatting ---
\titleformat{\section}{\Large\bfseries\color{blue!70!black}}{\thesection}{1em}{}
\titleformat{\subsection}{\large\bfseries\color{blue!50!black}}{\thesubsection}{1em}{}

\begin{document}

% --- PAGE 1: TITLE PAGE ---
\begin{titlepage}
    \centering
    \vspace*{2in}
    \Huge \textbf{Dynamic Systems Analysis} \\
    \vspace{0.4in}
    \Large \textit{Mathematical Foundations and Case Study of Second-Order LTI Systems} \\
    \vspace{1.5in}
    \large Prepared by: Engineering Research Division \\
    \vfill
    \large \today
\end{titlepage}

% --- PAGE 2: ABSTRACT (SEPARATE PAGE) ---
\newpage
\section*{Abstract}
\addcontentsline{toc}{section}{Abstract}
This technical investigation provides an exhaustive analytical exploration into the behavior, modeling, and frequency-domain characteristics of Linear Time-Invariant (LTI) systems. The primary focus is the mathematical rigor of second-order dynamics and the utility of the Laplace Transform as a bridge between time-domain differential equations and complex frequency-domain algebraic representations. This document provides a rigorous derivation of the system transfer functions, stability criteria, and the physics of damped harmonic motion. 

The scope extends beyond theoretical mathematics into mechanical engineering applications, specifically utilizing a "Quarter-Car" suspension model to simulate real-world road-vehicle interactions. We analyze the impact of mass, damping, and stiffness coefficients on the system's transient response and steady-state stability. Furthermore, this study delves into the phenomena of resonance and frequency attenuation, providing engineers with the predictive tools necessary to optimize mechanical performance. By integrating computational simulations in Python and traditional analytical proofs, this report serves as a complete pedagogical and professional reference for dynamic systems analysis in modern engineering contexts.
\vfill
\newpage

% --- PAGE 3: TABLE OF CONTENTS (SEPARATE PAGE) ---
\tableofcontents
\newpage

% --- PAGE 4: THEORY ---
\section{Nomenclature and Integral Theory}
The analysis utilizes standard mechanical and control theory notation to describe the interaction between energy storage and dissipation elements:
\begin{description}[leftmargin=1.5cm, style=nextline]
    \item[$m, M$] Sprung mass representing vehicle body weight (kg).
    \item[$c$] Damping coefficient of the hydraulic shock absorber (N·s/m).
    \item[$k$] Spring stiffness constant (N/m).
    \item[$s$] Complex frequency variable ($\sigma + j\omega$).
    \item[$\zeta$] Damping ratio, characterizing the decay rate of oscillations.
    \item[$\omega_n$] Undamped natural frequency (rad/s).
\end{description}

\subsection{Differential Property Proof}

Applying the definition $F(s) = \int_{0}^{\infty} f(t) e^{-st} dt$, we use integration by parts for the second derivative, which is essential for Newton's second law. This transformation converts the differential equation of motion into an algebraic characteristic equation, allowing for the isolation of the displacement variable $X(s)$ relative to the input force $F(s)$.

\section{Second-Order System Model}


[Image of a mass-spring-damper system diagram]

The governing equation $m\ddot{x} + c\dot{x} + kx = F(t)$ is transformed into the canonical transfer function. This model represents the physical balance between inertial force ($m\ddot{x}$), damping force ($c\dot{x}$), and spring restoration ($kx$).
\[ H(s) = \frac{\omega_n^2}{s^2 + 2\zeta\omega_n s + \omega_n^2} \]
The stability of the system is evaluated by the locations of the poles. If the poles are in the left-half of the $s$-plane, the system is inherently stable.

\newpage

% --- PAGE 5: CENTERED CASE STUDY TITLE (SEPARATE PAGE) ---
\newpage
\vspace*{\fill}
\begin{center}
    \Huge \textbf{CASE STUDY:} \\
    \huge \textbf{Quarter-Car Suspension Model Analysis}
\end{center}
\vspace*{\fill}
\newpage

% --- PAGE 6: CASE STUDY CONTENT & FUTURE WORK ---
\section{Application and Resonance Analysis}
In this case study, we evaluate a luxury sedan system configured with a sprung mass $m=450$ kg, a stiffness $k=28,000$ N/m, and a damping coefficient $c=2,500$ N·s/m. The resulting damping ratio $\zeta \approx 0.35$ confirms an underdamped response, which is the industry standard for passenger comfort as it allows for a soft absorption of road energy.

\subsection{Resonance Phenomenon}

Resonance occurs when the input frequency matches $\omega_r = \omega_n\sqrt{1-2\zeta^2}$. At this critical frequency, the system's gain is maximized, meaning the cabin will oscillate at a higher amplitude than the road bump itself. Analyzing this peak is vital to prevent mechanical fatigue and passenger discomfort.

\section{Future Work: Active Control Systems}
Passive suspension is limited by its fixed components. Future work will investigate Active Suspension using Proportional-Integral-Derivative (PID) control. By utilizing sensors to measure vertical acceleration and displacement, an actuator can apply a counter-force to keep the vehicle level.
\[ C(s) = K_p + \frac{K_i}{s} + K_d s \]


\section{Conclusion}
The Laplace Transform remains the cornerstone of modern mechanical design. By optimizing the damping ratio $\zeta$, engineers can effectively balance the trade-offs between ride quality and vehicle handling. This study confirms that while passive systems provide a reliable baseline, the integration of active feedback loops is necessary for high-performance applications where variable road conditions require real-time adaptation.
\newpage

% --- PAGE 7: DEEP APPENDIX A & REFERENCES (ON SAME PAGE) ---
\section{Appendix A: Detailed Numerical Simulation and Analysis}
This appendix provides the full computational framework used to validate the mathematical models. The simulation uses the \texttt{scipy.signal} library to solve the LTI system numerically.

\subsection{Step Response Verification}

The following script calculates the system's reaction to a 10cm curb impact. It computes the peak overshoot ($M_p$) and the 2\% settling time ($t_s$), which are critical metrics for vehicle safety standards.

\begin{lstlisting}[language=Python]
import numpy as np
import matplotlib.pyplot as plt
from scipy import signal

# Quarter-Car System Parameters
m = 450.0    # kg
k = 28000.0  # N/m
c = 2500.0   # Ns/m

# Transfer Function Definition
omega_n = np.sqrt(k/m)
zeta = c / (2 * np.sqrt(m * k))
num = [omega_n**2]
den = [1, 2*zeta*omega_n, omega_n**2]
sys = signal.TransferFunction(num, den)

# Time-Domain Simulation
t = np.linspace(0, 5, 1000)
t, y = signal.step(sys, T=t)

# Plotting the Results
plt.figure(figsize=(10, 5))
plt.plot(t, y, 'b', label='Displacement (m)')
plt.title('Vehicle Step Response - Curb Impact')
plt.xlabel('Time (s)')
plt.ylabel('Vertical Displacement')
plt.grid(True)
plt.show()

# Peak Overshoot and 2% Settling Time
M_p = np.max(y) - 1.0
t_s = np.where(y > 1.02)[0][0] / 1000.0
\end{lstlisting}
\subsection{Interpretation of Results}
The simulation confirms that with $\zeta=0.35$, the system exhibits an overshoot. This demonstrates that the shock absorbers are tuned for luxury; a higher $c$ value would reduce this bounce but increase the harshness felt by the passenger.

\vspace{0.01cm}
% --- REFERENCES (NO NEW PAGE) ---
\addcontentsline{toc}{section}{References}
\bibliographystyle{plain}
\bibliography{references}
\nocite{*}

\end{document}